\documentclass{beamer}
\usepackage[utf8]{inputenc}
\usepackage{amsmath}
\usepackage{amsfonts}
\usepackage{amssymb}
\usepackage{graphicx}
\usepackage{hyperref}
\hypersetup{
        colorlinks=true,
        linkcolor=blue,
        filecolor=magenta,
        urlcolor=cyan,
    }
\title{Graduate-Level Course: XVA in Financial Markets}
\author{Manus AI}
\date{\today}

\begin{document}

\begin{frame}
    \titlepage
\end{frame}


\begin{frame}
    \section*{Module 1: Introduction to XVA}
    \frametitle{Module 1: Introduction to XVA}

\end{frame}

\begin{frame}
    \subsection*{Topic 1.1: What is XVA?}
    \frametitle{Topic 1.1: What is XVA?}

    XVA stands for "X-Value Adjustment," a collective term for various valuation adjustments applied to the fair value of derivative transactions. These adjustments go beyond the traditional risk-neutral valuation and account for real-world factors such as counterparty credit risk, funding costs, and regulatory capital. The primary XVA components include:

    \begin{itemize}
    \item **CVA (Credit Valuation Adjustment):** The cost of hedging the counterparty credit risk of a derivative. It represents the expected loss due to the counterparty defaulting.
    \item **DVA (Debit Valuation Adjustment):** The benefit arising from the bank\'s own credit risk. It represents the expected gain if the bank itself defaults.
    \item **FVA (Funding Valuation Adjustment):** The cost or benefit associated with funding the uncollateralized portion of a derivative transaction.
    \item **MVA (Margin Valuation Adjustment):** The cost of funding the initial margin (IM) required for centrally cleared or bilateral derivative transactions.
    \item **KVA (Capital Valuation Adjustment):** The cost of holding regulatory capital against derivative exposures.
    \end{itemize}

    \framesubtitle{Why is XVA Important?}

    Before the 2008 financial crisis, derivative pricing primarily focused on risk-neutral valuation, which assumes no counterparty default risk and perfect funding. However, the crisis highlighted the significant impact of these real-world factors on derivative portfolios. XVA has become crucial for several reasons:

    \begin{itemize}
    \item **Accurate Pricing:** XVA provides a more accurate and comprehensive valuation of derivatives by incorporating all relevant costs and risks.
    \item **Risk Management:** It allows financial institutions to better understand, measure, and manage the various risks associated with their derivative portfolios.
    \item **Regulatory Compliance:** Regulators (e.g., Basel Committee) have introduced rules that mandate the calculation and reporting of XVA, particularly CVA and KVA.
    \item **Capital Allocation:** XVA influences capital allocation decisions, as higher XVA charges can lead to higher capital requirements.
    \end{itemize}

\end{frame}

\begin{frame}
    \subsection*{Topic 1.2: Historical Context and Evolution of XVA}
    \frametitle{Topic 1.2: Historical Context and Evolution of XVA}

    The concept of XVA has evolved significantly over time, driven by market events, regulatory changes, and advancements in financial modeling.

    \begin{itemize}
    \item **Pre-2008 Crisis:** Derivative pricing was largely based on the Black-Scholes-Merton framework, which assumes no default risk. Credit risk was typically managed through credit limits and collateral agreements, but not explicitly priced into the derivative itself.
    \item **Post-2008 Crisis (Emergence of CVA):** The collapse of Lehman Brothers and the subsequent credit crisis exposed the massive losses incurred by financial institutions due to counterparty defaults. This led to the widespread adoption of CVA as a critical component of derivative pricing.
    \item **DVA Controversy:** DVA emerged as the mirror image of CVA. However, its inclusion in derivative valuation sparked controversy, as it implies that a bank\'s financial health improves when its own creditworthiness deteriorates. Despite the debate, DVA is generally accepted under accounting standards (e.g., IFRS 13).
    \item **FVA and Funding Costs:** As central bank liquidity became less abundant and funding markets became more segmented, the cost of funding derivative positions became a significant factor. FVA was introduced to account for these funding costs.
    \item **MVA and Initial Margin:** The introduction of mandatory initial margin requirements for non-centrally cleared derivatives (e.g., under BCBS-IOSCO framework) led to the development of MVA to capture the funding cost of this margin.
    \item **KVA and Regulatory Capital:** Post-crisis regulations, particularly Basel III and IV, increased capital requirements for banks. KVA emerged to reflect the cost of holding this regulatory capital against derivative exposures.
    \end{itemize}

\end{frame}

\begin{frame}
    \subsection*{Topic 1.3: Overview of Derivatives and Markets}
    \frametitle{Topic 1.3: Overview of Derivatives and Markets}

    To understand XVA, it\'s essential to have a solid grasp of derivative instruments and the markets in which they trade.

    \begin{itemize}
    \item **Definition of Derivatives:** Financial contracts whose value is derived from an underlying asset, index, or rate. Common types include forwards, futures, options, and swaps.
    \item **OTC vs. Exchange-Traded Derivatives:**
    \end{itemize}
        *   **Over-the-Counter (OTC) Derivatives:** Customized contracts negotiated directly between two parties. They carry counterparty credit risk and are the primary focus of XVA.
        *   **Exchange-Traded Derivatives:** Standardized contracts traded on organized exchanges. They are typically centrally cleared, which significantly reduces counterparty credit risk.
    \begin{itemize}
    \item **Key Characteristics of Derivatives:** Notional amount, maturity, underlying asset, payment frequency, and collateral arrangements.
    \end{itemize}

\end{frame}

\begin{frame}
    \subsection*{Topic 1.4: Limitations of Risk-Neutral Pricing}
    \frametitle{Topic 1.4: Limitations of Risk-Neutral Pricing}

    Risk-Neutral pricing is a cornerstone of modern financial theory, but it has limitations when applied to real-world derivative markets, especially in the context of XVA.

    \begin{itemize}
    \item **Assumptions of Risk-Neutral Pricing:**
    \end{itemize}
        *   No default risk (perfect creditworthiness of all parties).
        *   Perfect funding (ability to borrow and lend at the risk-free rate).
        *   No transaction costs or taxes.
        *   Perfect liquidity.
    \begin{itemize}
    \item **Breakdown of Assumptions:** In reality, these assumptions do not hold. Counterparties can and do default, funding costs vary, and regulatory capital is required. XVA aims to bridge the gap between theoretical risk-neutral pricing and real-world valuation.
    \end{itemize}

\end{frame}

\begin{frame}
    \subsection*{Topic 1.5: The Interplay of XVA Components}
    \frametitle{Topic 1.5: The Interplay of XVA Components}

    XVA components interact in complex ways, and a holistic view is required to avoid double-counting or mispricing. Netting agreements, collateralization, and funding strategies affect multiple adjustments simultaneously.

    \begin{itemize}
    \item **CVA and DVA:** Represent bilateral credit risk. Improvements in one counterparty's credit quality reduce CVA but increase the other's DVA.
    \item **FVA and Collateral:** Collateral posted under credit support annexes reduces uncollateralized exposure and hence FVA, but the type of collateral influences funding costs.
    \item **MVA and Funding:** Initial margin funding costs feed into both MVA and broader funding considerations captured by FVA.
    \item **KVA Interactions:** The capital charge depends on exposures that already include CVA, DVA, and FVA effects, linking regulatory capital to other XVAs.
    \end{itemize}

\end{frame}

\begin{frame}
    \section*{Module 2: Credit Valuation Adjustment (CVA)}
    \frametitle{Module 2: Credit Valuation Adjustment (CVA)}

\end{frame}

\begin{frame}
    \subsection*{Topic 2.1: Definition and Rationale of CVA}
    \frametitle{Topic 2.1: Definition and Rationale of CVA}

    Credit Valuation Adjustment (CVA) is the most prominent XVA component. It represents the market value of the credit risk of a counterparty in a derivative transaction. In simpler terms, it\'s the amount a bank charges a client (or sets aside) to cover the potential loss if the client defaults on their obligations.

    \framesubtitle{Why CVA is Necessary}

    Traditional derivative pricing assumes that both parties to a contract will honor their obligations. However, in reality, there\'s always a risk that a counterparty might default. If a counterparty defaults when the derivative has a positive mark-to-market (MTM) value for the bank, the bank incurs a loss. CVA quantifies this expected loss and adjusts the derivative\'s price accordingly.

    CVA can be thought of as the price of a credit default swap (CDS) on the counterparty, where the protection buyer is the bank and the protection seller is the market.

\end{frame}

\begin{frame}
    \subsection*{Topic 2.2: Mathematical Formulation of CVA}
    \frametitle{Topic 2.2: Mathematical Formulation of CVA}

    CVA is typically calculated as the Expected Exposure (EE) weighted by the Probability of Default (PD) and Loss Given Default (LGD) over the life of the derivative. The general formula for CVA is:

    CVA = LGD * EPE

    Where:

    \begin{itemize}
    \item **LGD (Loss Given Default):** The percentage of the exposure that is lost if a default occurs. It is typically (1 - Recovery Rate).
    \item **EPE (Expected Positive Exposure):** The weighted average of the expected exposure (EE) over the life of the transaction, where the weights are the probabilities of default. More precisely, it\'s the average of the positive expected exposure, as the bank only incurs a loss if the counterparty defaults when the bank has a positive MTM.
    \end{itemize}

    \framesubtitle{Components of CVA Calculation}

    1.  **Exposure at Default (EAD):** The value of the derivative at the time of default. This is often approximated by the Mark-to-Market (MTM) value. However, for future dates, the MTM is uncertain, so Expected Exposure (EE) is used.
    2.  **Expected Exposure (EE):** The expected value of the exposure at a future point in time. Since the MTM of a derivative can be positive or negative, EE considers only the positive values (from the bank\'s perspective) as these are the only ones that lead to a loss upon counterparty default.
        *   EE is typically calculated using Monte Carlo simulations, where future market parameters (interest rates, exchange rates, etc.) are simulated, and the derivative is revalued at each simulation path.
    3.  **Probability of Default (PD):** The likelihood that the counterparty will default over a specific period. PDs are usually derived from credit spreads (e.g., from CDS markets) or historical default data.
    4.  **Loss Given Default (LGD):** The proportion of the exposure that is lost if a default occurs. It is usually expressed as 1 minus the recovery rate (RR).

    The discrete form of CVA can be expressed as:

    CVA = Sum[i=1 to T] (EE\_i * PD\_i * LGD * DF\_i)

    Where:

    \begin{itemize}
    \item `EE\_i`: Expected Exposure at time `t\_i`
    \item `PD\_i`: Probability of default in the interval `(t\_\{i-1\}, t\_i]`
    \item `LGD`: Loss Given Default
    \item `DF\_i`: Discount Factor to time `t\_i`
    \end{itemize}

\end{frame}

\begin{frame}
    \subsection*{Topic 2.3: Practical Aspects of CVA Calculation}
    \frametitle{Topic 2.3: Practical Aspects of CVA Calculation}

    Calculating CVA in practice involves several complexities:

    \begin{itemize}
    \item **Monte Carlo Simulation:** For portfolios of derivatives, especially those with complex payoff structures or multiple underlying risk factors, Monte Carlo simulation is the standard approach to generate future exposure paths.
    \item **Credit Spreads:** Deriving PDs from credit spreads requires careful consideration of market liquidity and calibration techniques.
    \item **Netting Agreements:** Most derivative master agreements (e.g., ISDA Master Agreement) include netting provisions, which allow for the aggregation of positive and negative MTMs across all transactions with a single counterparty. This significantly reduces the net exposure and, consequently, the CVA.
    \item **Collateral Agreements:** Collateral (e.g., cash, securities) posted by counterparties further reduces exposure. CVA calculations must account for collateral amounts, thresholds, minimum transfer amounts, and dispute resolution mechanisms.
    \item **Wrong-Way Risk:** This occurs when the exposure to a counterparty is adversely correlated with the counterparty\'s credit quality. For example, if a bank\'s exposure to a counterparty increases when the counterparty\'s probability of default also increases. This is a significant challenge in CVA modeling and requires advanced techniques to capture.
    \item **Right-Way Risk:** The opposite of wrong-way risk, where exposure decreases as credit quality deteriorates.
    \end{itemize}

\end{frame}

\begin{frame}
    \subsection*{Topic 2.4: CVA Hedging and Management}
\begin{frame}
    \frametitle{CVA Desks and Their Mandate}

    CVA risk arises because counterparty credit spreads move with the market, creating P\&L volatility that must be actively managed.

    \begin{itemize}
        \item Price CVA into new trades so business lines internalize credit costs.
        \item Track exposures and CVA Greeks, rebalancing hedges as spreads move.
        \item Support stress testing, capital planning, and senior management reporting.
    \end{itemize}
\end{frame}

\begin{frame}
    \frametitle{Common Hedging Instruments}

    \begin{itemize}
        \item Single-name CDS to hedge specific counterparties.
        \item Index CDS (CDX, iTraxx) when single-name markets are illiquid.
        \item Options on CDS, credit-linked notes, and bond shorts for non-linear or maturity mismatches.
        \item Supplementary rate or FX derivatives when market factors drive exposure profiles.
    \end{itemize}
\end{frame}

\begin{frame}
    \frametitle{Challenges in CVA Hedging}

    \begin{itemize}
        \item Basis risk from reference entity mismatches and limited liquidity.
        \item Need for dynamic rebalancing amid spread jumps or outright defaults.
        \item Model uncertainty, hedge accounting frictions, and funding costs of protection.
    \end{itemize}
\end{frame}

\begin{frame}
    \frametitle{Governance and Communication}

    \begin{itemize}
        \item Governance frameworks set limits and document hedging rationale.
        \item Regular reporting translates results for management, auditors, and regulators.
        \item Coordination with collateral, treasury, and finance aligns pricing, funding, and accounting.
        \item Clear communication of CVA costs guides client negotiations and counterparty selection.
    \end{itemize}
\end{frame}

\begin{frame}[allowframebreaks]
    \subsection*{Topic 2.5: Regulatory Capital for CVA (Basel III)}
    \frametitle{Topic 2.5: Regulatory Capital for CVA (Basel III)}

    Under Basel III, regulators formally recognized that counterparty credit risk in derivatives is not fully captured by default risk alone. Instead, fluctuations in the \emph{market value} of counterparty risk---known as Credit Valuation Adjustment (CVA)---can drive significant volatility in a bank's profit and loss. To ensure that institutions remain resilient against these risks, Basel III introduced a dedicated \emph{capital charge for CVA risk}. This charge serves as a prudential buffer, requiring banks to hold additional capital against unexpected losses arising not from outright default alone, but from credit spread movements that affect the valuation of derivative exposures.

    The CVA capital framework covers two principal dimensions of risk. First, it addresses \emph{spread volatility}, reflecting the sensitivity of derivative valuations to movements in counterparty credit spreads. Even without a default event, widening spreads can reduce the value of derivative receivables, creating losses. Second, it incorporates \emph{jump-to-default} risk, which captures the possibility of sudden counterparty failure. Together, these components ensure that the capital charge is forward-looking, providing coverage against both gradual and abrupt credit shocks.

    From an implementation perspective, Basel III provides banks with two broad approaches:

    \begin{enumerate}
        \item \textbf{Standardized Approach (SA-CVA):} The standardized method applies prescribed formulas and regulatory risk weights to determine capital requirements. It is designed for smaller or less complex banks that may lack the resources to build sophisticated internal models. While simpler to apply, it is less sensitive to the nuances of a bank's actual counterparty exposures and hedging strategies.
        \item \textbf{Advanced Approach (IMA-CVA):} The advanced method allows banks, subject to supervisory approval, to employ internal models to capture their own exposure dynamics, hedging strategies, and portfolio diversification. This approach is more risk-sensitive and better aligned with economic reality but comes with higher model validation, governance, and data requirements. Only banks with strong quantitative infrastructure and robust risk management practices are permitted to adopt it.
    \end{enumerate}

    The prudential rationale behind these requirements is clear: CVA losses during the 2008 financial crisis were a major source of stress for global banks, with billions of dollars written down due to counterparty spread widening. Regulators concluded that capital frameworks that focused only on default events were insufficient. By assigning explicit capital to CVA risk, Basel III ensures that institutions internalize the cost of counterparty risk management, discouraging excessive build-up of unhedged derivative exposures.

    Beyond compliance, these rules have reshaped market practice. Many banks have established dedicated \emph{CVA desks} responsible for pricing, managing, and hedging CVA risk. The cost of capital linked to CVA now directly influences product pricing, client negotiations, and even strategic decisions about which counterparties to face. Furthermore, the choice between standardized and advanced approaches has competitive implications: banks using internal models may optimize capital more efficiently but must also bear the cost of maintaining sophisticated infrastructure.

    In sum, the Basel III framework for regulatory capital on CVA represents a critical step in aligning regulatory standards with market reality. It extends beyond traditional default risk, capturing the market-driven valuation swings that can undermine solvency. By doing so, it not only strengthens the resilience of individual banks but also enhances the stability of the global financial system.
\end{frame}

\begin{frame}
    \section*{Module 3: Debit Valuation Adjustment (DVA)}
    \frametitle{Module 3: Debit Valuation Adjustment (DVA)}

\end{frame}

\begin{frame}
    \subsection*{Topic 3.1: Definition and Rationale of DVA}
    \frametitle{Topic 3.1: Definition and Rationale of DVA}

    Debit Valuation Adjustment (DVA) is the mirror image of CVA. While CVA accounts for the risk of a counterparty defaulting on its obligations to the bank, DVA accounts for the risk of the bank itself defaulting on its obligations to the counterparty.

    \framesubtitle{Why DVA is Included}

    Under fair value accounting principles (e.g., IFRS 13), financial liabilities are valued at the price that would be paid to transfer the liability to a third party. If a bank\'s own creditworthiness deteriorates, the cost of transferring its liabilities (including negative MTM derivative positions) decreases. This reduction in the value of the bank\'s liabilities is recognized as a gain, hence DVA is a positive adjustment to the derivative\'s value.

\end{frame}

\begin{frame}
    \subsection*{Topic 3.2: Mathematical Formulation of DVA}
    \frametitle{Topic 3.2: Mathematical Formulation of DVA}

    The mathematical formulation of DVA is analogous to CVA, but it considers the bank\'s own probability of default and the expected negative exposure (from the bank\'s perspective).

    DVA = LGD\_bank * ENE

    Where:

    \begin{itemize}
    \item **LGD\_bank (Loss Given Default of the bank):** The percentage of the exposure that is lost if a default occurs.
    \item **ENE (Expected Negative Exposure):** The weighted average of the expected negative exposure over the life of the transaction, where the weights are the probabilities of the bank\'s default. The bank only gains if it defaults when the derivative has a negative MTM for the bank (i.e., the bank owes money to the counterparty).
    \end{itemize}

    The discrete form of DVA can be expressed as:

    DVA = Sum[i=1 to T] (ENE\_i * PD\_bank\_i * LGD\_bank * DF\_i)

    Where:

    \begin{itemize}
    \item `ENE\_i`: Expected Negative Exposure at time `t\_i`
    \item `PD\_bank\_i`: Probability of default of the bank in the interval `(t\_\{i-1\}, t\_i]`
    \item `LGD\_bank`: Loss Given Default of the bank
    \item `DF\_i`: Discount Factor to time `t\_i`
    \end{itemize}

\end{frame}

\begin{frame}
    \subsection*{Topic 3.3: Controversies and Accounting Treatment of DVA}
    \frametitle{Topic 3.3: Controversies and Accounting Treatment of DVA}

    DVA has been a subject of significant debate and controversy, primarily because it implies that a bank benefits financially from its own deteriorating credit quality. This counter-intuitive outcome has led to public and political scrutiny.

    \begin{itemize}
    \item **Accounting Standards:** Despite the controversy, DVA is generally required under fair value accounting standards like IFRS 13 (Fair Value Measurement) and US GAAP (ASC 820). These standards mandate that the fair value of a liability should reflect the non-performance risk of that liability, which includes the entity\'s own credit risk.
    \item **Pro-cyclicality:** A major criticism of DVA is its pro-cyclical nature. During times of financial stress, a bank\'s credit spread widens, leading to an increase in DVA (a gain), which can artificially boost earnings when the bank is already in distress. This can create a misleading picture of the bank\'s financial health.
    \item **Hedging DVA:** Hedging DVA is challenging and often not undertaken by banks. Hedging a DVA gain would typically involve buying protection on the bank\'s own debt, which is often seen as sending a negative signal to the market about the bank\'s creditworthiness.
    \end{itemize}

\end{frame}

\begin{frame}
    \section*{Module 4: Funding Valuation Adjustment (FVA)}
\frametitle{Module 4: Funding Valuation Adjustment (FVA)}

\end{frame}

\begin{frame}
    \subsection*{Topic 4.1: Definition and Rationale of FVA}
    \frametitle{Topic 4.1: Definition and Rationale of FVA}

    Funding Valuation Adjustment captures the present value of liquidity spreads applied to unsecured derivative exposure. When the mark-to-market $V(t)$ is positive, the dealer must borrow at a spread above the overnight indexed swap (OIS) curve; negative exposure delivers a benefit that is typically less generous than OIS. Collateral thresholds, liquidity buffers, and contingency funding policies therefore determine how aggressively FVA accrues over the life of a trade.

    \begin{itemize}
    \item Positive unsecured exposure $E^{+}(t) = \max(V(t), 0)$ requires funding at the treasury curve.
    \item Negative exposure $E^{-}(t) = \max(-V(t), 0)$ yields an investing benefit bounded by reinvestment bases.
    \item FVA complements CVA/DVA by embedding real-world liquidity frictions in transfer pricing.
    \end{itemize}

\end{frame}

\begin{frame}
    \subsection*{Topic 4.2: Mathematical Formulation of FVA}
    \frametitle{Topic 4.2: Mathematical Formulation of FVA}

    \begin{align*}
        \text{FVA}(t_0) &= \int_0^T \mathbb{E}^{\mathbb{Q}}\big[ s(t)E^{+}(t) - b(t)E^{-}(t) \big] D_d(t)\, dt, \\
        \text{FVA} &\approx -\sum_{j=1}^{M} \Delta t_j D_{d,j} s_j \bar{E}^{+}_j + \sum_{j=1}^{M} \Delta t_j D_{d,j} b_j \bar{E}^{-}_j.
    \end{align*}

    \begin{itemize}
    \item $s(t) = r_f(t) - r_d(t)$ is the unsecured funding spread; $b(t) = r_d(t) - r_i(t)$ captures reinvestment bases.
    \item $D_d(t)$ denotes discount factors derived from the OIS curve used for collateralised valuation.
    \item Monte Carlo engines produce scenario-averaged exposures $\bar{E}^{\pm}_j$ aligned with treasury funding term structures.
    \item Wrong-way risk links counterparty stress to widening funding spreads, requiring joint simulation of credit and liquidity factors.
    \end{itemize}

\end{frame}

\begin{frame}
    \subsection*{Topic 4.3: Collateral, Discounting, and Infrastructure}
    \frametitle{Topic 4.3: Collateral, Discounting, and Infrastructure}

    \begin{itemize}
    \item \textbf{Thresholds and Minimum Transfer Amounts:} Determine how much exposure remains unsecured before collateral moves.
    \item \textbf{Rehypothecation Rights:} If collateral cannot be reused, the investing benefit term $b(t)E^{-}(t)$ collapses.
    \item \textbf{Discount-Curve Coordination:} FVA must remain consistent with the curves used for CVA/DVA to avoid double counting.
    \item \textbf{Data Lineage:} Exposure cubes, funding curves, and stress scenarios require shared identifiers and governance.
    \end{itemize}

\end{frame}

\begin{frame}
    \subsection*{Topic 4.4: Debate, Hedging, and Management}
    \frametitle{Topic 4.4: Debate, Hedging, and Management}

    The economic interpretation of FVA remains debated. Proponents view it as essential to internalise scarce liquidity; critics warn of double counting relative to credit spreads. Governance frameworks document modelling assumptions, stress scenario design, and accounting treatment to manage this debate.

    \begin{itemize}
    \item \textbf{Funding Strategy:} Mix of unsecured debt, repos, and collateral transformation to stabilise $s(t)$.
    \item \textbf{Client Terms:} Negotiating thresholds, eligible collateral, and clearing choices to minimise expected exposure.
    \item \textbf{Monitoring:} Dashboards track sensitivities to spreads, collateral assumptions, and wrong-way risk correlations.
    \end{itemize}

\end{frame}

\begin{frame}
    \section*{Module 5: Margin Valuation Adjustment (MVA)}
    \frametitle{Module 5: Margin Valuation Adjustment (MVA)}

\end{frame}

\begin{frame}
    \subsection*{Topic 5.1: Definition and Rationale of MVA}
    \frametitle{Topic 5.1: Definition and Rationale of MVA}

    Margin Valuation Adjustment quantifies the opportunity cost of posting segregated initial margin (IM) for cleared and bilateral portfolios. IM is bankruptcy-remote and often remunerated below the dealer's funding curve, creating a persistent funding wedge that is sensitive to volatility regimes, concentration limits, and collateral eligibility schedules.

    \begin{itemize}
    \item Regulatory mandates under BCBS--IOSCO expand the scope of trades requiring IM.
    \item IM balances are path dependent because they respond to portfolio composition and market volatility.
    \item Treasury attributes a funds-transfer-pricing spread to support segregated collateral over the trade horizon.
    \end{itemize}

\end{frame}

\begin{frame}
    \subsection*{Topic 5.2: Mathematical Formulation of MVA}
    \frametitle{Topic 5.2: Mathematical Formulation of MVA}

    \begin{align*}
        \text{MVA} &= \mathbb{E}^{\mathbb{Q}}\left[ \int_0^T D(t) IM(t) \big(s_f(t) - r_{rem}(t)\big) \mathbf{1}_{\{t < \tau_c \wedge \tau_b\}} dt \right], \\
        \text{MVA} &\approx \sum_{k=1}^{n} IM(t_k) \big(s_f(t_k) - r_{rem}(t_k)\big) D(t_k) \Delta t.
    \end{align*}

    \begin{itemize}
    \item $IM(t)$ is derived from the institution's initial margin model applied to Monte Carlo exposure cubes.
    \item $s_f(t)$ captures funds-transfer-pricing spreads; $r_{rem}(t)$ is the custodian remuneration rate.
    \item Integration stops at the first default time $\tau_c \wedge \tau_b$, reflecting close-out mechanics.
    \item Wrong-way risk links funding spreads and margin surges under stress scenarios.
    \end{itemize}

\end{frame}

\begin{frame}
    \subsection*{Topic 5.3: Initial Margin Models}
    \frametitle{Topic 5.3: Initial Margin Models}

    \begin{itemize}
    \item \textbf{ISDA SIMM:} Sensitivity-based approach with prescribed buckets, risk weights, and correlations; transparent but potentially conservative for bespoke portfolios.
    \item \textbf{Internal VaR/ES Models:} Simulate potential future exposure over liquidation horizons, capturing diversification but demanding rigorous back-testing and supervisory approval.
    \item \textbf{Hybrid Governance:} Institutions reconcile standardized and internal results via overrides, buffers, and model-risk controls.
    \end{itemize}

\end{frame}

\begin{frame}
    \subsection*{Topic 5.4: Hedging and Management}
    \frametitle{Topic 5.4: Hedging and Management}

    \begin{itemize}
    \item \textbf{Collateral Optimisation:} Portfolio compression, clearing mandates, and collateral transformation to reduce IM balances.
    \item \textbf{Funding Strategy:} Term repos, committed liquidity facilities, and structured notes to lock in spreads before volatility spikes.
    \item \textbf{Transfer Pricing:} Treasury dashboards attribute funding costs back to trading desks, promoting margin-efficient deal design.
    \end{itemize}

\end{frame}

\begin{frame}
    \subsection*{Topic 5.5: Regulatory Impact}
    \frametitle{Topic 5.5: Regulatory Impact}

    \begin{itemize}
    \item \textbf{BCBS--IOSCO Governance:} Phased implementation, eligible collateral schedules, and dispute resolution expectations.
    \item \textbf{Operational Investment:} Margin analytics, reconciliation tooling, and legal documentation embedded into pricing.
    \item \textbf{Stress Testing:} Supervisors scrutinise scenarios where volatility-driven IM surges coincide with widening funding spreads.
    \end{itemize}

\end{frame}

\begin{frame}
    \section*{Module 6: Advanced Topics and Integrated XVA}
    \frametitle{Module 6: Advanced Topics and Integrated XVA}

\end{frame}

\begin{frame}
    \subsection*{Topic 6.1: Interdependencies of XVA Components}
    \frametitle{Topic 6.1: Interdependencies of XVA Components}

    \begin{align*}
        \text{XVA}_{\text{total}} &= \text{CVA} - \text{DVA} + \text{FVA} + \text{KVA} + \text{MVA}, \\
        \text{CVA} &= (1 - R_c) \int_0^T EPE(t) \lambda_c(t) S_c(t) D(0,t)\, dt, \\
        \text{FVA} &= \int_0^T EPE(t) [f(t) - r(t)] D(0,t)\, dt.
    \end{align*}

    \begin{itemize}
    \item Collateral choices shift expected exposure profiles, altering credit, funding, and capital metrics simultaneously.
    \item Scenario generation must remain synchronised across credit, funding, and margin models to avoid inconsistent adjustments.
    \item Hedging decisions are evaluated on their joint impact across CVA, FVA, MVA, and KVA.
    \end{itemize}

\end{frame}

\begin{frame}
    \subsection*{Topic 6.2: XVA Desks and Organisational Structure}
    \frametitle{Topic 6.2: XVA Desks and Organisational Structure}

    \begin{itemize}
    \item \textbf{Integrated Operating Model:} Centralised XVA desks align front office pricing with treasury funding and capital costs.
    \item \textbf{Governance:} Risk management validates methodologies, finance ensures accounting alignment, and treasury manages liquidity.
    \item \textbf{Data Infrastructure:} Golden-source identifiers and real-time data feeds support fast recalibration when markets shift.
    \end{itemize}

\end{frame}

\begin{frame}
    \subsection*{Topic 6.3: Regulatory Landscape and Future of XVA}
    \frametitle{Topic 6.3: Regulatory Landscape and Future of XVA}

    \begin{align*}
        \text{SA-CVA} &= \sqrt{\sum_i K_i^2 + 2 \sum_{i<j} \rho_{ij} K_i K_j}, \\
        \text{Stress Capital Buffer} &= \frac{\max(\text{Projected Loss} - \text{Allowances}, 0)}{\text{Risk-Weighted Assets}}.
    \end{align*}

    \begin{itemize}
    \item FRTB-CVA emphasises risk-sensitive capital metrics and robust scenario generation.
    \item Supervisors expect transparent model governance, documented data lineage, and climate-aware stress scenarios.
    \item Technological innovation (cloud, AI, DLT) must be paired with explainability and operational resilience.
    \end{itemize}

\end{frame}

\begin{frame}
    \subsection*{Topic 6.4: Case Studies and Real-World Examples}
    \frametitle{Topic 6.4: Case Studies and Real-World Examples}

    \begin{align*}
        \text{Hedge Effectiveness} &= 1 - \frac{\Delta \text{Unhedged CVA}}{\Delta \text{Gross CVA}}, \\
        \text{Capital Impact Ratio} &= \frac{\text{Post-Remediation CVA Capital}}{\text{Pre-Remediation CVA Capital}}.
    \end{align*}

    \begin{itemize}
    \item Case studies examine funding shocks, collateral disputes, and remediation programmes.
    \item Metrics quantify whether hedges, process upgrades, or data investments delivered expected economic relief.
    \item Lessons inform governance enhancements and client communication strategies.
    \end{itemize}

\end{frame}

\begin{frame}
    \subsection*{Topic 6.5: Future Trends and Challenges in XVA}
    \frametitle{Topic 6.5: Future Trends and Challenges in XVA}

    \begin{itemize}
    \item \textbf{Climate and ESG Analytics:} Integrating transition and physical risk scenarios into valuation adjustments.
    \item \textbf{Technology Enablement:} AI-driven exposure approximations and DLT-based collateral management with strong model-risk controls.
    \item \textbf{Operational Resilience:} Cybersecurity, data quality, and senior-management accountability as core supervisory themes.
    \end{itemize}

\end{frame}
\section*{References}
    \frametitle{References}

    \begin{itemize}
    \item[1] Corporate Finance Institute. "XVA (X-Value Adjustment) - Definition, Types, Component." Available at: \url{https://corporatefinanceinstitute.com/resources/valuation/xva-x-value-adjustment/}
    \item[2] Risk.net. "Debit valuation adjustment (DVA) definition." Available at: \url{https://www.risk.net/definition/debit-valuation-adjustment-dva}
    \item[3] Corporate Finance Institute. "Credit Valuation Adjustment (CVA) - Overview, Formula, History." Available at: \url{https://corporatefinanceinstitute.com/resources/derivatives/credit-valuation-adjustment-cva/}
    \item[4] CQF. "What is a Funding Value Adjustment?" Available at: \url{https://www.cqf.com/blog/quant-finance-101/what-is-a-funding-value-adjustment}
    \item[5] Risk.net. "Capital valuation adjustment (KVA) definition." Available at: \url{https://www.risk.net/definition/capital-valuation-adjustment-kva}
    \item[6] Risk.net. "Margin valuation adjustment (MVA) definition." Available at: \url{https://www.net/definition/margin-valuation-adjustment-mva}
    \end{itemize}

\end{frame}

\end{document}


