\section{End of April Module 6: Capital Valuation Adjustment (KVA)}

\subsection*{Spoken Introduction}
Capital Valuation Adjustment is the mechanism that forces us to price the cost
of regulatory capital directly into trades. When a dealer originates a new
transaction, the balance sheet must hold risk-weighted assets, leverage ratio
buffers, and stress capital. Those buffers are not free: shareholders expect a
return on the equity that is locked in, and regulators penalise us if capital
falls short. KVA translates that reality into a stream of costs that we
subtract from deal economics. In practice the adjustment is smaller than CVA or
FVA on vanilla portfolios, but it becomes decisive for structured or
concentrated trades where capital density is high.

\subsection{Regulatory Capital Stack}
To compute KVA we start by projecting the capital that regulators demand. Pillar~1
charges capture default risk, CVA risk, and counterparty credit exposures.
Pillar~2 add-ons absorb model limitations, governance weaknesses, or
concentration risks. Macro buffers such as the countercyclical or systemic risk
buffers sit on top. Each layer is expressed either as a percentage of
risk-weighted assets or as a leverage constraint on total exposure. We map the
portfolio to these frameworks using internal models or the standardised
approach, reconcile with finance, and then forecast how capital evolves under
business plans and stress scenarios.

\subsection{KVA Formula and Interpretation}
Once the capital profile is available, we apply the hurdle rate that management
requires for holding that capital. The discrete approximation mirrors the
formula shown on the slide:
\begin{equation*}
\mathrm{KVA} = \sum_{k} \mathrm{Capital}(t_k)\,\bigl(c - r(t_k)\bigr)\,D(0,t_k)\,\Delta t.
\end{equation*}
Here $c$ is the target return on equity, $r(t_k)$ is the risk-free discount rate
used for pricing, and $D(0,t_k)$ discounts the net cost back to today. Positive
differences between the hurdle and risk-free rates create a cost; if
regulations remunerated capital at the hurdle rate there would be no KVA.

\subsection{Managing the Cost}
Because KVA cannot be hedged in the market, we manage it structurally. We
favour centrally cleared trades, optimise collateral agreements, and compress
back-to-back trades to shrink the exposure-at-default that feeds capital. We
also align client pricing with the capital profile, so desks feel the charge
when they quote. Strategic initiatives—such as securitising portfolios or
transferring exposures to entities with cheaper capital—may be justified when
KVA dominates deal economics.

\subsection{Interdependencies}
Changes that reduce CVA or FVA can still increase KVA if they drive up risk-
weighted assets. For example, shortening funding tenors might lower FVA but
push the leverage ratio higher, demanding more capital. That is why we run
integrated scenario analysis across all XVA components. The steering committee
reviews those scenarios before approving model changes or large trades, making
sure KVA sits alongside CVA, DVA, FVA, and MVA in every decision.
