\section{Module 6: Capital Valuation Adjustment (KVA)}

\subsection{6.1 Conceptual Foundations of KVA}
Capital Valuation Adjustment (KVA) captures the opportunity cost of holding
regulatory capital against derivative exposures. Whereas CVA, DVA, FVA, and
MVA focus on expected losses, funding frictions, or margin segmentation, KVA
links valuation to the return-on-equity targets that shareholders require on
regulatory capital. Basel III and the Fundamental Review of the Trading Book
(FRTB) specify capital buffers for counterparty credit risk, default risk, and
market risk. KVA internalises these costs by discounting the stream of capital
charges using the institution's hurdle rate. Without KVA, trade pricing can
appear profitable while eroding risk-adjusted returns because capital is locked
in to absorb potential losses.

\subsection{6.2 Regulatory Capital Stack and Measurement}
Regulatory capital requirements arise from multiple layers: minimum Pillar~1
charges for counterparty credit risk and CVA risk, Pillar~2 add-ons for model
and governance deficiencies, and macro-prudential buffers such as the
countercyclical capital buffer. Banks aggregate risk-weighted assets (RWA) from
standardised or internal models and translate them into capital using target
ratios. For an internal models approach (IMA) desk, CVA capital might be
computed as
\begin{equation}
K^{\text{CVA}}_{\text{IMA}} = \sqrt{\mathrm{VaR}^2_{10\text{d}} + \mathrm{SVaR}^2_{10\text{d}}}
\times \mathrm{MC},
\end{equation}
where $\mathrm{VaR}$ and $\mathrm{SVaR}$ are ten-day measures calibrated to
current and stressed periods, and $\mathrm{MC}$ denotes a multiplier reflecting
model quality. Default risk capital, exposure-at-default calculations, and
leverage ratio constraints also feed into the capital stack that KVA must fund.

\subsection{6.3 KVA Quantification and Discounting}
KVA is commonly approximated by projecting required capital balances and
charging them at the bank's cost of capital net of any return recognised in the
valuation. A discrete-time formulation is
\begin{equation}
\mathrm{KVA} = \sum_{k=1}^{n} \mathrm{Capital}(t_k)\,\bigl( c - r(t_k) \bigr)\,D(0,t_k)\,\Delta t,
\end{equation}
where $c$ is the hurdle rate, $r(t_k)$ is the risk-free rate consistent with the
valuation framework, and $D(0,t_k)$ is the discount factor. Capital balances may
be obtained from projected risk-weighted assets, stressed exposure profiles, or
credit spread scenarios. Because capital metrics are typically reported
quarterly, practitioners interpolate between reporting dates while preserving the
impact of step-changes triggered by portfolio rebalancing or model approvals.

\subsection{6.4 Governance, Hedging, and Strategic Responses}
Unlike CVA or FVA, KVA cannot be hedged directly in financial markets; instead,
institutions optimise capital usage through portfolio steering, collateral
negotiations, and product design. Frequent measures include
\begin{itemize}
  \item prioritising centrally cleared trades or secured financing that attract
  lower risk weights;
  \item compressing redundant trades to reduce exposure-at-default and
  counterparty concentration add-ons;
  \item issuing capital-efficient structures, such as structured notes matched
  to client demand, that recycle capital-intensive risks;
  \item embedding capital charges into funds-transfer-pricing so desks internalise
  the cost when quoting trades.
\end{itemize}
Governance frameworks assign ownership for capital projections, align treasury
and risk appetite statements, and document escalation procedures when capital
buffers erode. Supervisors expect transparent reconciliation between pricing
adjustments, regulatory filings, and management reporting.

\subsection{6.5 Interaction with Other XVA Components}
KVA interacts with the broader XVA stack through shared exposure scenarios and
policy choices. Tightening collateral terms reduces expected exposure and CVA,
but it may increase initial margin and therefore MVA; the resulting change in
risk-weighted assets flows into KVA. Funding strategies that shorten liability
maturities can lower FVA but raise leverage ratio constraints, adding to KVA.
Stress testing needs to propagate shocks to credit spreads, exposures, and
capital simultaneously so that management decisions consider aggregate economic
value. Integrating KVA into dashboards and profitability metrics ensures that
trading desks balance headline revenues with risk-adjusted returns.
