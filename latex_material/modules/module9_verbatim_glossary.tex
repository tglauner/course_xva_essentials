\section{Module 9: Glossary and Key Terms}

\subsection{9.1 Pillars of XVA}
This section compiles the foundational valuation adjustments referenced throughout the course so readers can quickly locate concise definitions and interpretive notes.

\begin{itemize}
    \item \textbf{Credit Valuation Adjustment (CVA):} Expected loss on positive exposures from counterparty default, blending exposure forecasts, default probabilities, and loss-given-default assumptions.
    \item \textbf{Debit Valuation Adjustment (DVA):} Valuation uplift that recognises the bank's own default risk and the potential reduction of its liabilities upon failure.
    \item \textbf{Funding Valuation Adjustment (FVA):} Cost of financing hedge cashflows at spreads above the risk-free or OIS curve, most acute on uncollateralised exposures.
    \item \textbf{Margin Valuation Adjustment (MVA):} Funding drag created by segregated initial margin that cannot be rehypothecated for cleared or bilateral derivatives.
    \item \textbf{Capital Valuation Adjustment (KVA):} Cost of holding regulatory capital over a trade's life, typically priced at the institution's hurdle rate to reflect shareholder return targets.
\end{itemize}

\subsection{9.2 Supporting Terms and Risk Metrics}
This section consolidates the broader credit, funding, and risk management vocabulary referenced throughout the course so the glossary remains aligned with the web content.
\begin{itemize}
    \item \textbf{Hurdle Rate:} Minimum return on equity a bank targets when deploying capital; it sets the discount rate for KVA and anchors risk-adjusted performance metrics used in governance and pricing.
    \item \textbf{Central Counterparty (CCP):} Clearing house that steps between trade counterparties, novates positions, collects margin (both initial and variation), and mutualises residual losses through a default fund and its own capital.
    \item \textbf{Loss Given Default (LGD):} Percentage of the exposure that is lost if a default occurs, typically expressed as one minus the recovery rate.
    \item \textbf{Expected Positive Exposure (EPE):} Weighted average of expected exposure over the transaction's life, considering only positive mark-to-market values.
    \item \textbf{Expected Negative Exposure (ENE):} Weighted average of expected negative exposure over the transaction's life, focusing on downside mark-to-market values.
    \item \textbf{Probability of Default (PD):} Likelihood that a counterparty will default over a specified period, often inferred from credit spreads or historical data.
    \item \textbf{Expected Exposure (EE):} Expected value of the exposure at a future point in time, considering only positive mark-to-market values, commonly generated via Monte Carlo simulation.
    \item \textbf{Overnight Index Swap (OIS):} Interest rate swap whose floating leg references an overnight index; used as the discounting curve for collateralised derivatives.
    \item \textbf{ISDA Standard Initial Margin Model (ISDA SIMM):} Standardised methodology for calculating initial margin for non-centrally cleared derivatives based on risk sensitivities.
    \item \textbf{Wrong-Way Risk:} Situation where exposure to a counterparty increases when the counterparty's credit quality deteriorates, implying adverse correlation between market and credit risk factors.
    \item \textbf{Right-Way Risk:} Situation where exposure decreases as the counterparty's credit quality worsens, reflecting beneficial correlation between exposure and creditworthiness.
    \item \textbf{Basel III:} International regulatory framework for banks that sets capital charges, including specific requirements for CVA risk.
    \item \textbf{Credit Default Swap (CDS):} Derivative contract that transfers credit risk of a reference entity; the premium reflects default probability, LGD, and notional exposure.
\end{itemize}
